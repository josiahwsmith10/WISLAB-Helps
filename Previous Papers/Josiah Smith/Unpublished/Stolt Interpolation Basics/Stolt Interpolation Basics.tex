\documentclass{IEEEtran}
\usepackage{amsmath}
\usepackage{amsfonts}
\usepackage{amssymb}
\usepackage{blindtext}
\usepackage{listings}

\usepackage{xcolor}
\definecolor{codegreen}{rgb}{0,0.6,0}
\definecolor{codegray}{rgb}{0.5,0.5,0.5}
\definecolor{codepurple}{rgb}{0.58,0,0.82}
\definecolor{backcolor}{rgb}{0.95,0.95,0.92}

\usepackage{color} %red, green, blue, yellow, cyan, magenta, black, white
\definecolor{mygreen}{RGB}{28,172,0} % color values Red, Green, Blue
\definecolor{mylilas}{RGB}{170,55,241}	
	
\lstdefinestyle{mystyle}{
	language=Matlab,%
	basicstyle=\ttfamily\footnotesize,
	backgroundcolor=\color{backcolor},
	breaklines=true,
	commentstyle=\color{mygreen},%
	showstringspaces=false,%without this there will be a symbol in the places where there is a space
	numbers=left,%
	numberstyle={\tiny \color{black}},% size of the numbers
	numbersep=1pt, % this defines how far the numbers are from the text 
	tabsize=4,
	captionpos=b
}

\usepackage{titlesec}
\titleformat{\section}[block]{\bfseries}{\thesection.}{1em}{}

\lstset{style=mystyle}

\title{Stolt Interpolation Basics with MATLAB Algorithms}

% Use letters for affiliations, numbers to show equal authorship (if applicable) and to indicate the corresponding author
\author{Josiah W. Smith}


\begin{document}

\maketitle

\begin{abstract}

Stolt interpolation, also known as range migration algorithm (RMA) or $\omega-k$ or $f-k$ migration, for the rectilinear and cylindrical cases in 2D and 3D using monostatic radar systems for near-field 2D and 3D image reconstruction.

\end {abstract}

\section{Introduction}
This paper is organized as follows. Section II will discuss the application of Stolt interpolation to a 1D-rectilinear SAR scanner using a single monostatic, SISO radar array for 2D image reconstruction in the $(x,z)$ domains and provide applicable MATLAB algorithms. Section III will extend the application of Stolt interpolation for use on a 2D-rectilinear SAR scanner using a single monostatic, SISO radar array for 3D holographic image reconstruction in the $(x,y,z)$ domains and provide necessary MATLAB algorithms. Section IV will propose and demonstrate a 1D-cylindrical SAR (CSAR) system and the proper Stolt interpolation to recover a 2D image with relevant MATLAB algorithms. And finally, Section V will discuss the extension of this CSAR system to a 2D-CSAR system for generating high resolution 3D holographic images and provide MATLAB algorithms.

\section{1D-Rectilinear SISO-SAR System Model for 2D Image Reconstruction by Stolt Interpolation}
The one-dimensional rectilinear single-input,single-output synthetic aperture radar (SISO-SAR) system consists of a single transceiver moved to evenly-spaced positions across the horizontal X-domain (can we get a figure?).

The received signal from a monostatic, full-duplex transceiver is of the form given by equation (1).

\begin{equation}
	s(x,k) = \iint \frac{p(x,z)}{R^2} e^{j2kR}dxdz
\end{equation}
\begin{equation}
	R = \sqrt{ (x-x')^2 + (z-z')^2 }
\end{equation}
Where $R$ is the distance from the transceiver to the target scene, $p(x,z)$ is the reflectivity function of the target scene, and $(x',z')$ is the location of the transceiver.

It has been shown in the literature that $p(x,z)$ can be recovered via Fourier-based techniques and circular to planar wavefront decomposition as described in equation (3).

\begin{equation}
	p(x,z) = IFT^{(k_x,k_z)}_{2D} \left[ Stolt^{(k)} \left( FT^{(x)}_{1D} \left[ s^*(x,z) \right] \right) \right]
\end{equation}
Where $FT$ and $IFT$ are the forward and backward Fourier transform operators, $(\cdot)^*$ is the complex conjugate operator, and $Stolt(\cdot)$ is the Stolt interpolation algorithm.

This is the fundamental equation for rectilinear Stolt interpolation and will be further examined in this paper.

\subsection{1D-Rectilinear SISO-SAR 2D Image Reconstruction and Necessity of Stolt Interpolation}
The single-input-single-output synthetic aperture radar (SISO-SAR) reconstruction algorithm is described by equations (4),(5),(6). Where equation (5) is the Stolt interpolation step, migrating $(k_x,k)$ to $(k_x,k_z)$ by the relation given in equation (7), derived from the dispersion relation of electromagnetic (EM) waves.

\begin{gather}
	S(k_x,k) = FT_{(x)}[s^*(x,k)] \\
	\hat{S}(k_x,k_z) = Stolt^{(k)}[S(k_x,k)] = S(k_x,k)\biggr\rvert^{k = \frac{1}{2} \sqrt{k_x^2 + k_z^2}} \\
	p(x,z) = IFT_{(k_x,k_z)}[\hat{S}(k_x,k_z)]\\
	(2k)^2 = k_x^2 + k_z^2
\end{gather}

Without the Stolt interpolation step in equation (5), the 2D image is not resolvable in the $(x,z)$ plane.

\subsection{MATLAB Algorithm for 2D-Image Reconstruction using Stolt Interpolation}

\lstinputlisting{StoltInterpolation2Dk_kz.m}

\section{2D-Rectilinear SISO-SAR System Model for 3D Holographic Image Reconstruction with Stolt Interpolation}
The system model for 3D image reconstruction is as follows. First, equation (8) is the received signal model.

\begin{equation}
s(y,x,k) = \iiint \frac{p(y,x,z)}{R^2} e^{-j2kR}dxdydz
\end{equation}
\begin{equation}
R = \sqrt{(x-x')^2 + (y-y')^2 + (z-z')^2}
\end{equation}

By similar wavefront decomposition and Fourier analysis, the 3D holographic image reconstruction algorithm can be consolidated by equation (10).

\begin{equation}
p(y,x,z) = IFT^{(k_y,k_x,k_z)}_{3D} \left[Stolt^{(k)}\left( FT^{(y,x)}_{2D} \left[ s^*(y,x,z) \right] \right) \right]
\end{equation}

\subsection{2D-Rectilinear SISO-SAR 3D-Image Reconstruction and Implementation of Stolt Interpolation}
Expanding the derivation provided in section II, the single-input-single-output synthetic aperture radar (SISO-SAR) reconstruction algorithm is described by equations (4),(5),(6). Where equation (5) is the Stolt interpolation step, migrating $(k_x,k)$ to $(k_x,k_z)$ by the relation given in equation (7), derived from the dispersion relation of electromagnetic (EM) waves.

\begin{equation}
S(k_y,k_x,k) = FT_{(y,x)}[s^*(y,x,k)]
\end{equation}
\begin{equation}
\begin{split}
	\hat{S}(k_y,k_x,k_z) & = Stolt[S(k_y,k_x,k)] \\
	& = S(k_y,k_x,k)\biggr\rvert^{k =\frac{1}{2} \sqrt{k_y^2 + k_x^2 + k_z^2}}
\end{split}
\end{equation}
\begin{gather}
p(y,x,z) = IFT_{(k_y,k_x,k_z)}[\hat{S}(k_y,k_x,k_z)] \\
(2k)^2 = k_y^2 + k_x^2 + k_z^2
\end{gather}

Without the Stolt interpolation step in equation (12), the 3D image is not resolvable in the $(y,x,z)$ plane.

\subsection{MATLAB Algorithm for 3D-Image Reconstruction using Stolt Interpolation}

\lstinputlisting{StoltInterpolation3Dk_kz.m}

\section{1D-Cylindrical SISO-CSAR System Model for 2D Image Reconstruction by Stolt Interpolation}
Using the same coordinate system described in sections II-III and the basic near-field cylindrical setup, the return signal can be modeled by equation (15).

\begin{gather}
	s(\theta,k) = \iint \frac{p(x,z)}{R^2}e^{j2kR}dxdz \\
	R = \sqrt{(R_0 cos\theta - x)^2 + (R_0 sin\theta - z)^2}
\end{gather}
Where $R_0$ is the radial distance from the center of the rotator to the radar transceiver.

\subsection{1D-Cylidrical SISO-CSAR 2D-Image Reconstruction and Implementation of Stolt Interpolation}
The algorithm for recovering the reflectivity function of the target scene is described by the following derivation.

First, the exponential term in equation (15, $s(\theta,k)$) can be approximated by:

\begin{multline}
	e^{j2k\sqrt{(R_0 cos\theta - x)^2 + (R_0 sin\theta - z)^2}} \approx \\ \int_{-\pi}^{\pi} e^{j2k(cos\alpha(R_0cos\theta-x)+sin\alpha(R_0sin\theta - z))} d\alpha
\end{multline}

Substituting into equation (15, $s(\theta,k)$) neglecting amplitude terms.

\begin{multline}
	s(\theta,k) = \iint p(x,z) \\ 
	\times \int_{-\pi}^{\pi} e^{j2k(cos\alpha(R_0cos\theta-x)+sin\alpha(R_0sin\theta - z))} d\alpha dx dz
\end{multline}
\begin{multline}
	s(\theta,k) = \int_{-\pi}^{\pi} \iint p(x,z) e^{-j2k(xcos\alpha+ysin\alpha)}dxdz \\
	\times e^{j2kR_0cos(\theta-\alpha)}d\alpha
\end{multline}

Defining the Fourier relation:

\begin{equation}
	p(\alpha,k) \triangleq \iint p(x,z) e^{-j2k(xcos\alpha+ysin\alpha)}dxdz
\end{equation}

Now

\begin{equation}
	s(\theta,k) = \int_{-\pi}^{\pi} p(\alpha,k) e^{j2kR_0cos(\theta-\alpha)}d\alpha
\end{equation}

It is obvious that equation (21) represents a circular convolution with respect to $\theta$, denoted by $\circledast_{\theta}$.

\begin{gather}
	h(\theta,k) = e^{j2kR_0cos\theta}\\
	s(\theta,k) = p(\theta,k) \circledast_{\theta} h(\theta,k)
\end{gather}

Taking the Fourier transform on both sides of equation (23) allows for multiplication in the Fourier domain to replace convolution spatial angular. Note that $\theta$ and $\Theta$ are conjugate variables of the Fourier transform.

\begin{gather}
	S(\Theta,k) = P(\Theta,k)H(\Theta,k) \\
	P(\Theta,k) = \frac{S(\Theta,k)}{H(\Theta,k)} \\
	p(\theta,k) = IFT_{1D}^{(\Theta)}\left[ \frac{S(\Theta,k)}{H(\Theta,k)} \right]
\end{gather}

By the Fourier relation in equation (20), $p(x,z)$ can be recovered from $p(\theta,k)$ by:

\begin{equation}
	p(x,z) \triangleq \iint p(\theta,k) e^{-j2k(xcos\theta+ysin\theta)}d\theta dk
\end{equation}

\section{2D-Cylindrical SISO-CSAR System Model for 3D Holographic Image Reconstruction by Stolt Interpolation}


\subsection{2D-Cylidrical SISO-CSAR 3D-Image Reconstruction and Implementation of Stolt Interpolation}

The received signal is of the form given by equation (15).
\begin{equation}
s(\theta,y,k) = \iiint \frac{p(x,y,z)}{R^2} e^{-j2kR}dxdydz
\end{equation}
\begin{equation}
R = \sqrt{(R_0 cos\theta-x)^2 + (R_0 sin\theta-y)^2 + (z' - z)^2}
\end{equation}
Where $R_0$ is the scanning radius, $\theta$ is the scanning angle, and $y'$ is the scanning height, with the target and scanning domains coinciding. $p(x,y,z)$ is the complex reflectivity function of the target scene.

Inverting equation (15) to solve for $p(x,y,z)$ requires decomposing the exponential term representing a spherical-wave into a superposition of plane-wave components using some Fourier-based techniques described in [REF].

Perform a 2D Fourier transform across the $\theta$ and $z$ dimensions.
\begin{equation}
S(\Theta,k,k_z) = FT^{(\theta,z)}_{2D}[s(\theta,k,z)]
\end{equation} 
Compute the phase term described in equation (18).
\begin{equation}
k_\Theta = \sqrt{4(k_x^2+k_y^2)R_0^2 - \Theta^2}
\end{equation}
Multiply the phase term and perform an inverse Fourier transform across the $\Theta$ domain.
\begin{equation}
\hat{P}(\theta,k,k_z) = IFT^{(\Theta)}_{1D} [S(\Theta,k,k_z)e^{-jk_\Theta}]
\end{equation}
Make the following substitutions:

\begin{equation}
P(k_x,k_y,k_z) = \hat{P}(\theta,k,k_z) \biggr\rvert^{\theta = tan^{-1}(\frac{k_y}{k_x}),k = \frac{1}{2} \sqrt{k_x^2+k_y^2+k_z^2}}
\end{equation}

\end{document}